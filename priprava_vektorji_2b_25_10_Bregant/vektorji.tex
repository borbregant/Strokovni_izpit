\documentclass{article}
\usepackage[utf8]{inputenc}
\usepackage{pgfplots}
\pgfplotsset{width=10cm,compat=1.9}
\usepackage{amsmath,amssymb,amsthm}
\usepackage{graphicx}
\usepackage{float}
\usepackage{blindtext}
\usepackage{hyperref}
\usepackage{verbatim}
\usepackage{gensymb}
\usepackage{enumerate}
\usepackage{xcolor}
\usepackage{graphicx}
\hypersetup{
    colorlinks=true,
    linkcolor=blue,
    filecolor=magenta,      
    urlcolor=cyan,
    pdftitle={Overleaf Example},
    pdfpagemode=FullScreen,
    }
\usepackage[slovene]{babel}

\setlength{\parindent}{0pt}
\setlength{\parskip}{4pt}

\newcounter{example}[section]
\newenvironment{example}[1][]{\refstepcounter{example}\par\medskip
   \noindent \textbf{Naloga~\theexample. #1} \rmfamily}{\medskip}

\newtheorem*{zgled}{Zgled}

\title{Vektorji}
\author{Bor Bregant}
\date{\vspace{-5ex}}

\begin{document}

\thispagestyle{empty}	% ne oštevilči strani

\noindent MATEMATIKA, \quad 2. B \hfill Škofijska klasična gimnazija
\hrule
\vspace{1ex}
\noindent \textbf{Tema:} Množenje vektorja s skalarjem
\vspace{1ex}

\noindent \textbf{Enota:} Vektorji
\vspace{1ex}

\noindent \textbf{Datum:} 25. 10. 2023
\vspace{1ex}

\noindent \textbf{Mentorica:} dr. Marina Rugelj
\vspace{1ex}

\noindent \textbf{Viri in literatura:} Planum novum, 2020, Pavlič G. in drugi
\vspace{1ex}
\hrule
\vspace{2ex}
\noindent \textbf{Učne oblike:} Frontalna, individualna
\vspace{1ex}

\noindent \textbf{Učne metode:} Metoda razprave v uvodu, razlaga
\vspace{1ex}

\noindent \textbf{Učni pripomočki:} Tabla, učbenik
\vspace{1ex}

\noindent \textbf{Učni cilji:} Dijaki/dijakinje znajo množiti vektorje s skalarjem na grafičnem nivoju
\vspace{4ex}
\hrule
\vspace{5ex}
\noindent \textbf{Vsebina in potek:} 

\vspace{5ex}

\section*{\textcolor{violet}{Vžig in uvod}}

Po pozdravu pregledamo morebitna vprašanja glede domače naloge. Včerajšnjo obravnavano snov na kratko preverimo s klicanjem posameznih dijakov, da povejo osnovna dejstva o vektorjih. \textcolor{violet}{Dijaki pri pogovoru sodelujejo in odgovarjajo.}

Začnemo z uvodom množenja vektorjev s številom v razpravi. Razmišljamo, kaj bi se zgodilo, če seštevamo enaka vektorja. \textcolor{violet}{Dijaki pri pogovoru sodelujejo.}


\section*{\textcolor{violet}{Razlaga snovi}}

\textbf{Množenje vektorja s številom (skalarjem)}

$m \cdot \vec{a}$, kjer $m\in\mathbb{R}_{\neq 0}$ je vektor z enako smerjo kot $\vec{a}$, usmerjenost je odvisna od predznaka $m$, njegova velikost pa je enaka $|m| \cdot |\vec{a}|$. Za $|m|>1$, se $\vec{a}$ podaljša, za $0<|m|<1$, se $\vec{a}$ skrajša.

Velja:
\begin{enumerate}[i]
    \item Asociativnost v skalarnem delju: $m(n\vec{a})=(mn)\vec{a}$,
    \item distributivnost v vektorskem delu: $m(\vec{a}+\vec{b})=m\vec{a}+m\vec{b}$,
    \item distributivnost v skalarnem delu: $(m+n)\vec{a}=m\vec{a}+n\vec{a}$.
\end{enumerate}

Iz vektorja $\vec{a}$ lahko naredimo enotski vektor $\vec{e}=\frac{\vec{a}}{|\vec{a}|}$.

\section*{\textcolor{violet}{Utrjevanje}}
\textbf{\textcolor{violet}{Primere naredimo skupaj.}}

\begin{zgled}
    Izračunaj $n$, če za vektor $\vec{a}$ velja $(n^2 +3)\vec{a}-(n+2)\vec{a}=(3+n)\vec{a}+\vec{a}$.
\end{zgled}

\begin{zgled}
    Nariši trikotnik $ABC$, kjer $|\vec{AB}|=6cm$, $|\vec{BC}|=5cm$ in $|\vec{CA}|=3cm$. Nato nariši vektor $\vec{a}+2\vec{c}$.
\end{zgled}

\textbf{\textcolor{violet}{Dijaki si DN zabeležijo in odidejo iz razreda}}

\begin{example}
    Domača naloga 286b, 293ab
\end{example}

\end{document}
