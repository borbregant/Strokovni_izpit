\documentclass{article}
\usepackage[utf8]{inputenc}
\usepackage{pgfplots}
\pgfplotsset{width=10cm,compat=1.9}
\usepackage{amsmath,amssymb,amsthm}
\usepackage{graphicx}
\usepackage{float}
\usepackage{blindtext}
\usepackage{hyperref}
\usepackage{verbatim}
\usepackage{gensymb}
\usepackage{enumerate}
\usepackage{xcolor}
\usepackage{graphicx}
\hypersetup{
    colorlinks=true,
    linkcolor=blue,
    filecolor=magenta,      
    urlcolor=cyan,
    pdftitle={Overleaf Example},
    pdfpagemode=FullScreen,
    }
\usepackage[slovene]{babel}

\setlength{\parindent}{0pt}
\setlength{\parskip}{4pt}

\newcounter{example}[section]
\newenvironment{example}[1][]{\refstepcounter{example}\par\medskip
   \noindent \textbf{Naloga~\theexample. #1} \rmfamily}{\medskip}

\newtheorem*{zgled}{Zgled}

\title{Funkcije}
\author{Bor Bregant}
\date{\vspace{-5ex}}

\begin{document}

\thispagestyle{empty}	% ne oštevilči strani

\noindent MATEMATIKA, \quad 4. A \hfill Škofijska klasična gimnazija
\hrule
\vspace{1ex}
\noindent \textbf{Tema:} Ponovitev že znanih pojmov o funkcijah, računanje s funkcijami
\vspace{1ex}

\noindent \textbf{Enota:} Funkcije
\vspace{1ex}

\noindent \textbf{Datum:} 17. 11. 2023
\vspace{1ex}

\noindent \textbf{Mentorica:} dr. Marina Rugelj
\vspace{1ex}

\noindent \textbf{Viri in literatura:} Tempus novum, 2020, Pavlič G. in drugi
\vspace{1ex}
\hrule
\vspace{2ex}
\noindent \textbf{Učne oblike:} Frontalna, individualna
\vspace{1ex}

\noindent \textbf{Učne metode:} Metoda razprave v uvodu, razlaga
\vspace{1ex}

\noindent \textbf{Učni pripomočki:} Tabla, učbenik
\vspace{1ex}

\noindent \textbf{Učni cilji:} Dijaki/dijakinje ponovijo in uporabljajo pojme, vezane na funkcije. Dijaki/dijakinje usvojijo računanje s funkcijami.
\vspace{4ex}
\hrule
\vspace{5ex}
\noindent \textbf{Vsebina in potek:} 

\vspace{5ex}

\section*{\textcolor{violet}{Uvodna motivacija}}

\textbf{\textcolor{violet}{V razpravi ponovimo že znane pojme in naloge pri funkcijah.}}


\textbf{\textcolor{violet}{Definicija, ki jo narekujemo}}

Funkcija ali preslikava $f$ iz množice $A$ v množico $B$ je predpis, ki vsakemu elementu iz $A$ priredi natanko en element iz množice $B$.

\textbf{\textcolor{violet}{Dijaki pred tablo rešijo naslednji sklop nalog}}

\begin{zgled}
    Zapiši definicijsko območje in ničle za funkcijo $f(x)=\sqrt{x^2-1}$ in $g(x)=\log{\left(x(x-1)(x+1)\right)}$.\\
    Ob grafu funkcije zapiši zalogo vrednosti intervale naraščanja in obravnavaj konveksnost za $h(x) = \cos\left(x-\frac{\pi}{2}\right)-4$.\\
    Obravnavaj sodost oziroma lihost za $m(x)=x^2 +x^6$ in $n(x)=x+x^3$.\\
    Zapiši ničle in začetno vrednost za $e(x)=2^{\frac{x}{2}}-\left(\frac{1}{2}\right)^4$.
\end{zgled}

\section{Računanje s funkcijami}

\begin{enumerate}[i]
    \item Vsota $\left(f+g\right)(x)=f(x)+g(x)$
    \item Razlika $\left(f-g\right)(x)=f(x)-g(x)$
    \item Produkt $\left(f\cdot g\right)(x)=f(x)\cdot g(x)$
    \item Kvocient $\left(\frac{f}{g}\right)(x)=\frac{f(x)}{g(x)}$ za $g(x)\neq 0, \forall x\in D_g$
    \item Produkt s številom $\left(k\cdot f\right)(x)=k \cdot f(x)$
\end{enumerate}

\section*{\textcolor{violet}{Utrjevanje}}
\textbf{\textcolor{violet}{Dijak rešuje nalogo pred tablo}}

\begin{zgled}
    Zapiši vsoto, razliko, produkt in kvocient funkcij $f(x)=x-3$ in $g(x)=-2x+1$.
\end{zgled}

\begin{example}
    Domača naloga 569c, 576, 578ad
\end{example}

\textbf{\textcolor{violet}{Dijaki si DN zabeležijo in odidejo iz razreda}}

\end{document}